\chapter{Basic Mathematics}
My first \LaTeX\ example with footnote for abc\footnote{means bac}. And this are my first formulae:
\begin{align}
  \SI{10}{\ohm} - \SI{9}{\ohm} &= \SI{1}{\ohm}\\
  % 10\Ohm -9\Ohm &= 1\Ohm % with myunits which must be activated in the header
  a^2 + b^2 &= c^2\,,\nonumber\\
  \leadsto c &= \sqrt{a^2+b^2}\label{eq:pythagoras}\,,\\
  a &= \operatorname{e}^{-\mathrm{i}\omega t}\,,\\
  a &= \frac{1}{2}\,,\\
  a_{n+1} &= a_n^2+\sum_{i=0}^{n}b^{2i}\,,\\
  \intertext{with text inbetween without changing the alignment of the equations}
  a_{n'+1}' &= a_{n'}^2+\sum_{i=0}^{n'}b^{2i}\,,\\
  \boldsymbol{b} &= \boldsymbol{a }+ \boldsymbol{c}\,,\\
  a &= a\cdot\, i\,,\qquad i\in\mathbb{N}\,.\\
  \int_a^b (x+x^2)\mathrm{d} x &= \int_a^b (x+x^2)\mathrm{d} x\\
  \oint_a^b (x+x^2)\mathrm{d}x &= \int_a^b (x+x^2)\mathrm{d}x\\
  \oiint_a^b (x+x^2)\mathrm{d}x &= \int_a^b (x+x^2)\mathrm{d}x\\
  \iiint_a^b (x+x^2)\mathrm{d}x &= \int_a^b (x+x^2)\mathrm{d}x\\
  \int\int\int_a^b (x+x^2)\mathrm{d}x &= \int_a^b (x+x^2)\mathrm{d}x\\
  \stackrel{\leftrightarrow}{A} &= \stackrel{\leftrightarrow}{B}+\stackrel{\leftrightarrow}{C}\\
  R_{\text{max}}& = \frac{T_{\mathrm{c}} \cdot c}{2 B} f_{\mathrm{max}}
\end{align}

$\pi$ is roughly\ldots

It's obvious, that Pythogoras' equation~(\ref{eq:pythagoras}) on page~\pageref{eq:pythagoras} is
wrong. The correct relation between the three lines in a rectangular triangle (without equation
number) is:
\begin{align*}
  c &= \sqrt[2.1]{a^2+b^2}\,.
\end{align*}

\begin{table}[!b]
  \centering
  \begin{tabular}{lr}
    Parameter                               & Value\\
    \hline
    carrier frequency $f_{\text{c}}$        & \SI{77}{\giga\hertz}\\
    bandwidth $B$                           & \SI{500}{\mega\hertz}\\
    chirp duration $T_{\text{c}}$           & \SI{20}{\micro\second}\\
    chirp repetition time $T_{\text{r}}$    & \SI{25}{\micro\second}\\
    sampling frequency $f_{\text{s}}$       & \SI{50}{\mega\hertz}\\
    number of chirps $L$                    & 128\\
    window function                         & Hann\\
    zero padding                            & signal length doubled
  \end{tabular}
  \caption{Specifications of Radar Parameters.}
  \label{tab:radar_para}
\end{table}

\clearpage % use a clearpage carefully and only if you know the sideeffects
\label{ici}
We are now on page number~\pageref{ici}.

\chapter{TikZ and PGFPlots example}
Example for the \emph{subcaption} package using a TikZ and PGFPlots image.

\begin{figure}[tbp]
  \centering
  \subcaptionbox{This is a TikZ image which can be scaled only with a fixed axis ratio.
  \label{fig:tikzimg}}
  [0.47\linewidth]{\includegraphics[width=0.47\linewidth]{gfx/chirp_sequence_shift_01.tikz}}
  \hfill
  \subcaptionbox{This is a PGFPlots example. Take a look at the source code for further examples.
    Using the tikzscale package the axis ratio can be adepted.
  \label{fig:pgfplotsimg}}
  [0.47\linewidth]{\includegraphics[width=0.47\linewidth,height=0.2\textheight]{gfx/example_plots.tikz}}
  \caption{This is an example for the subcaption package.}
  \label{fig:subcapex}
\end{figure}

If in image~\ref{fig:subcapex} a linewidth smaller than 0.49 is chosen, the images are positioned
left and right. If the size is larger as now, they are positioned differently. A reference to
Bild~\ref{fig:subcapex}~(\subref{fig:tikzimg}). Also note that the axis ratio for a PGFPlots image
can be influenced.

\begin{figure}[tbp]
  \centering
  \subcaptionbox{This is a TikZ image which can be scaled only with a fixed axis ratio.
  \label{fig:tikzimg2}}
  [0.60\linewidth]{\includegraphics[width=0.6\linewidth]{gfx/chirp_sequence_shift_01.tikz}}
  \par\bigskip
  \subcaptionbox{This is a PGFPlots example. Take a look at the source code for further examples.
    Using the tikzscale package the axis ratio can be adepted.
  \label{fig:pgfplotsimg2}}
  [0.60\linewidth]{\includegraphics[width=0.6\linewidth,height=0.35\textheight]{gfx/example_plots.tikz}}
  \caption{This is an example for the subcaption package.}
  \label{fig:subcapex2}
\end{figure}

The examples in Bild~\ref{fig:subcapex} and~\ref{fig:subcapex2} use either a TikZ image or a
PGFPlots image consisting of a function or data points. For large data matrices this approach is not
useful. Here use the \emph{graphics} command of PGFPlots. See~\ref{fig:pgfplots} for an example.

\begin{figure}[tbp]
  \begin{minipage}[b][\textheight][b]{\linewidth}
    \centering
    \subcaptionbox{This is an image which was created using the \emph{imagesc} plot function from
      Matlab in combination with matlab2tikz.
    \label{fig:imagesc}}
    [\linewidth]{\includegraphics[width=0.95\linewidth]{gfx/surfimgsc.tikz}}
    \vfill
    \subcaptionbox{For 3D plots there is no suitable way with matlab2tikz. Create the png on your own
      and create an axis environment with PGFPlots.
    \label{fig:surf2tikz3d}}
    [\linewidth]{\includegraphics[width=0.65\linewidth]{gfx/surf2tikz3D.tikz}}
    %
    \caption{An example for 2D and 3D plots.}
    \label{fig:pgfplots}
  \end{minipage}
\end{figure}

\begin{figure}[tbp]
  \centering
  \tikzset{%
	% Self defined bulding blocks. 
	% Nevertheless circutikz has implemented filters, couplers and other components since version 0.4, they are mostly implemented as bipoles.
	% The usage of bipoles: \draw (start) to[lowpass/amp/adc,....] (end).
	% The problem is, that if one wants to use arrows, the arrows in bipoles can not be sat manual (fixed in circuitikz source) AND THEY ARE NOT CONSISTENT
	% Also it's quite a mess, which component is a monopole, simple block, bipol, quad/triple etc
	% Following are a few examples on how to define your own blocks. 
	%
	% % % % % % % % % % % % % % % % % % % % % % % % % % % % % % % % % % % % % % % % % % % % % % % % % % % % % % % % % % % % % % % % % % % % %
	% % % % % % % % % % % % % % % % % % % % % % % % % % % % % % % % % % % % % % % % % % % % % % % % % % % % % % % % % % % % % % % % % % % % %
	% % % % % % % % % % % % % % % % % % % % % % % % % % % % % % % % % % % % % % % % % % % % % % % % % % % % % % % % % % % % % % % % % % % % %
	% % % % % % % % % % % % % % % % % % % % % % % % % % % % % % % % % % % % % % % % % % % % % % % % % % % % % % % % % % % % % % % % % % % % %
	%
	% Standard block definition, the width and height is adopted from the circutizk source code, so don't mind the strange values. Also the linewidth is set according to the circutrikz source code.
	block/.style    	= 	{draw, fill=white, thick, rectangle, minimum height = 0.98cm, minimum width = 0.98cm, node distance=2.5cm, line width=1.5pt},
	%
	% Standard circular block
	circleblock/.style	= 	{draw, fill=white, thick, circle, minimum width = 0.98cm,  line width=1.5pt, node distance=2.5cm},
	%
	% Label for circuitikz nodes, as they're reference is in the middle and not on the outer edge of the node....
	circuitikzlabel/.style	=	{label={[label, label distance=0.5cm]#1}},
	%
	%
	%
	% VCO/Oscillator 
	myVCO/.style			= 	{circleblock, path picture={%
		\draw[line width=0.75pt] 	($(path picture bounding box.west)+(0.09cm,0)$) sin ($(path picture bounding box.center)-(0.2cm,-0.2cm)$) cos  (path picture bounding box.center) sin ($(path picture bounding box.center)-(-0.2cm,0.2cm)$) cos ($(path picture bounding box.east)-(0.09cm,0)$);
		}
	},
	% Amplifier, as circuitikz does only provite amplifiers as 2-ports/bipoles
	myAMP/.style		= 	{block, node distance=2.5cm, path picture={%
		\draw[fill=white, line width=0.75pt] ($(path picture bounding box.center)+(0.7em,0)$) -- ($(path picture bounding box.center)-(0.7em,-0.7em)$) -- ($(path picture bounding box.center)-(0.7em,0.7em)$)  -- cycle;
		}
	},
	% Same for ADC
	myADC/.style 	=	{block, path picture={%
		\draw[line width=0.75pt] 	(path picture bounding box.south west) -- (path picture bounding box.north east);
		\node[] at ($(path picture bounding box.center)+(-.5em,.5em)$) () {D};
		\node[] at ($(path picture bounding box.center)+(.5em,-.5em)$) () {A};
		} 
	},
	% Same for filters
	myLP/.style	=	{block, path picture={%
		%Sine-Waves
		\draw[line width=.75pt] 	($(path picture bounding box.west)+(0.3em,0)$) sin ($(path picture bounding box.center)-(0.50em,-0.3em)$) cos  (path picture bounding box.center) sin ($(path picture bounding box.center)-(-0.50em,0.3em)$) cos ($(path picture bounding box.east)-(0.3em,0)$);
		\draw[line width=0.75pt] 	($(path picture bounding box.west)+(0.3em,-0.65em)$) sin ($(path picture bounding box.center)-(0.50em,0.35em)$) cos  ( $(path picture bounding box.center)-(0,0.65em)$) sin ($(path picture bounding box.center)-(-0.50em,0.95em)$) cos ($(path picture bounding box.east)-(0.3em,0.65em)$);
		\draw[line width=0.75pt] 	($(path picture bounding box.west)+(0.3em,0.65em)$) sin ($(path picture bounding box.center)-(0.50em,-0.95em)$) cos  ( $(path picture bounding box.center)+(0,0.65em)$) sin ($(path picture bounding box.center)-(-0.50em,-0.35em)$) cos ($(path picture bounding box.east)-(0.3em,-0.65em)$);
		% Cancelation
		\draw[line width=0.75pt] 	($(path picture bounding box.center)-(0.2em,0.2em)$) -- (path picture bounding box.center) -- ($(path picture bounding box.center)+(0.2em,0.2em)$) ;
		\draw[line width=0.75pt] 	($(path picture bounding box.center)-(0.2em,-0.45em)$) -- ($(path picture bounding box.center)+(0,0.65em)$) -- ($(path picture bounding box.center)+(0.2em,0.85em)$) ;
		}
	},
}
\begin{tikzpicture}[line width=0.7pt,>=latex,node distance=2.5cm]
	% First: All building blocks are placed relative to the first component
	\draw (0,0)
		node[myVCO, label={below:VCO}] (oszi) {}
		% As the coupler ports are not in the middle, based on the size (again extraceted from circutikz source code), an yshift is perfomed to have the input on the same height as the output of the oszillator. The xshift is used to place the VCO and ADC on the same y-value after all.
		node[coupler, right of=oszi, yshift=-0.455cm, xshift=1cm] (coupler) {}
		% Undo the yshift as the output of the coupler is on the same height as the input of the amplifier
		node[myAMP, right of=coupler, yshift=0.455cm, node distance=4cm, label={below:PA}] (pa) {}
		% Circulator is rotated that the ports are on the correct position, normally ports are arranged as follows:
		% 
		%				  o------------o
		%						|
		%						o
		node[circulator, below right of=pa, rotate=90] (circ) {}
		node[antenna, right of = circ] (antenna) {}
		node[myAMP, below left of=circ, rotate=180, label={LNA}] (lna) {}
		% Used redefinition of label, otherwise the label would be overlapping with the mixer shape
		node[mixer, left of=lna, circuitikzlabel={below:mixer}] (mixer) {}
		node[myLP, left of=mixer, label={below:lowpass}] (lowpass) {}
		node[myADC, left of=lowpass] (adc) {};
	
	% Connect everything together
	\draw[->] (oszi) -- (coupler.4);
	\draw[->] (coupler.3) -- (pa.west);
	% Match is placed relative to coupler port, yscale=-1 mirrors the component at the y-axis
	\draw[] (coupler.1) -| ++(-0.5cm,-0.1cm) node[match, rotate=-90, yscale=-1] {};
	\draw[->] (pa.east) -| (circ.2);
	\draw[-] (circ.3) -- (antenna);
	\draw[->] (circ.1) |- (lna.west);
	\draw[->] (lna.east) -- (mixer.east);
	\draw[->] (coupler.2) -| (mixer.north);
	\draw[->] (mixer.west)  -- (lowpass.east);
	\draw[->] (lowpass.west) -- (adc.east);
\end{tikzpicture}

  \caption{A \emph{circuitikz} example.}
  \label{fig:cwradar}
\end{figure}

\begin{figure}[tbp]
  \centering
  \includegraphics[width=\linewidth]{gfx/smith_chart.tikz}
  \caption{A Smith Chart example.}
  \label{fig:sc}
\end{figure}

\begin{figure}[tbp]
  \centering
  \includegraphics{gfx/fc_example.tikz}
  \caption{A flow chart example.}
  \label{fig:fc}
\end{figure}

If you use several TikZ oder PGFPlots consider to active the \emph{externalize} option. Then think about a
shell-escape.

\chapter{Include literature}
As an example a literature is included. Only entries which are referenced as like~\cite{Roos2015}
are printed.
