%%%%%%%%%%%%%%%%%%%%%%%%%%%%%%%%%%%%%%%%%%%%%%%%%%%%%%%%%%%%%%%%%%%%%%%%%%%%%%%%%%%%%%%%%%%%%%%%%%%%
%%% +++---~~~~ Anfang: Vorlage für Grafikexport für Speicherung als Bild ~~~---+++ %%%%%%%%%%%%%%%%%
%%%%%%%%%%%%%%%%%%%%%%%%%%%%%%%%%%%%%%%%%%%%%%%%%%%%%%%%%%%%%%%%%%%%%%%%%%%%%%%%%%%%%%%%%%%%%%%%%%%%
%%% Autor:		Fabian Roos, fabian@froos.de
%%% Datum:		20.12.2013 (Nicht Einstellungsänderung, sondern Inhalt!)
%%% Dokument:   Exportiere Grafiken als PDF, speichere als Bild (Vim: 'gg=G' erlaubt!)
%%%	            PDF zu Bild umwandeln: 'convert -density 300 export.pdf export.png'
%%%%%%%%%%%%%%%%%%%%%%%%%%%%%%%%%%%%%%%%%%%%%%%%%%%%%%%%%%%%%%%%%%%%%%%%%%%%%%%%%%%%%%%%%%%%%%%%%%%%


%%% +++---~~~ Anfang: documentclass-Definitionen ~~~---+++ %%%%%%%%%%%%%%%%%%%%%%%%%%%%%%%%%%%%%%%%%
% Siehe auch: Paketinformationen: http://www.ctan.org/tex-archive/macros/latex/contrib/standalone
\documentclass[%
% ~~~~~~~~~ Document ~~~~~~~~~~~~~~~~~~~~~~~~~~~~~~~~~~~~~~~~~~~~~~~~~~~~~~~~~~~~~~~~~~~~~~~~~~~~~~~
    class=scrreprt,		% Welche Dokomentklasse soll als Vorlage verwendet werden? S.8
                        % Vorteil: Verwende 'scrreprt' für Kompatibilität und Makros
                        %          wie '\footnotesize'
    tikz,				% Lädt das TikZ-Paket und setzt die Umgebungsvariablen. S. 7
    border=1pt,         % Fügt einen Rand ein, damit Pfeile nicht abgeschnitten werden
]{standalone}
%%% +++---~~~ Ende: documentclass-Definitionen ~~~---+++  %%%%%%%%%%%%%%%%%%%%%%%%%%%%%%%%%%%%%%%%%%

\newcommand{\LANGUAGE}{1}                       % 1 = german, 0 = english

%%% +++---~~~ Anfang: Pakete laden ~~~---+++ %%%%%%%%%%%%%%%%%%%%%%%%%%%%%%%%%%%%%%%%%%%%%%%%%%%%%%%
% ~~~~~~~~ Standard-Pakete ~~~~~~~~~~~~~~~~~~~~~~~~~~~~~~~~~~~~~~~~~~~~~~~~~~~~~~~~~~~~~~~~~~~~~~~~~
% Für die drei folgenden Pakete siehe auch: http://de.wikipedia.org/wiki/LaTeX#Zeichenkodierung
\usepackage[utf8]{inputenc}	% Zeichencodierung

\usepackage[T1]{fontenc}	% Schriftart, für Umlaute
\usepackage[ngerman]{babel}	% Für deutsche Begriffe (Inhaltsver., Kapitel) und Worttrennung

% ~~~~~~~~ Mathemathik-Pakete ~~~~~~~~~~~~~~~~~~~~~~~~~~~~~~~~~~~~~~~~~~~~~~~~~~~~~~~~~~~~~~~~~~~~~~
\usepackage{amsmath}		% Für Mathematikumgebungen wie equation, align (& vor =, S. 3)
                % Formeln referenzieren mit \eqref, siehe S. 10
                % beinhaltet laut Dokumentation S. 2:
                % -> amstext: \text innerhalb von Gleichungen
                % -> amsopn: Neue Formeln mit \DeclareMathOperator
                % -> amsbsy: Rückwärtskompatibilität
                % Bei Problemen \usepackage{mathtools}
%\usepackage{amssymb}		% Für besondere Zeichen, etwa \dashrightarrow, lädt
                % -> amsfont: Extra Fonts und Symbole
                % http://de.wikibooks.org/wiki/LaTeX-Kompendium:_amssymb
%\usepackage{esint}		% Spezielle Integrale
\usepackage[%                       % SIunits ist veraltet, siehe LaTeX2e-Sündenregister S. 15
                                    % Wie soll das Trennzeichen bei Zahlen aussehen?
    % decimalsymbol=comma,	        % Dezimaltrenner (Version 1)
    output-decimal-marker={,},      % Dezimaltrenner (Version 2)
    % expproduct = {\cdot},         % Produktzeichen bei 1 \cdot 10^-5 ist Standard x (Version 1)
    exponent-product={\cdot},       % Produktzeichen bei 1 \cdot 10^-5 ist Standard x (Version 2)
    per-mode=fraction,              % Ohne diese Angabe folgt m s^-1 statt als Bruch
    range-phrase={--},              % Für \SIrange{1}{10}{einheit} statt to ein -
    % range-phrase={ bis },         % Für \SIrange{1}{10}{einheit} statt to ein bis
    %range-units=single,            % Damit nicht nach jedem eine Einheit erscheint
    list-final-separator={ und },   % Für \SIlist{1;2;3}{\einheit}: Letzte Trennung wie?
    list-pair-separator={ und },    % Für \SIlist{1;2}{\einheit}: Trennung in der Mitte wie?
    % list-units=brackets,          % Damit bei SIlist (1;2;3) Einheit erscheint statt 1Einheit, 2Einheit
    detect-all,                     % Damit in strong-Umgebungen auch so dargstellt. Ggf. andere Option?
]{siunitx}                          % Beispiel: \SI{1,1}{\metre} kein '{,}' nötig

% ~~~~~~~~ Grafik-Pakete ~~~~~~~~~~~~~~~~~~~~~~~~~~~~~~~~~~~~~~~~~~~~~~~~~~~~~~~~~~~~~~~~~~~~~~~~~~~
% TikZ (auch in 'export*' benötigt - standalone-Paket-Datei)
\usepackage{tikz}		% Für Tikz-Grafiken
\usetikzlibrary{%
    external,		% Bibliothek für das externalisieren von TikZ
    calc,			% Für die Berechnung von Parametern
    shapes.geometric,	% Flussdiagramme
    arrows,			% Pfeile (irgendwann auf arrows.meta wechseln, wenn nativ vorhanden)
    fit,			% Flussdiagramme: Boxen um Nodes zeichnen
    trees,			% for 'circular growth' bei den Baumdiagrammen
    % snakes,       % superseded, 'decorations.pathmorphing' verwenden
    decorations.pathmorphing,
}
\usepackage{tkz-euclide}	% Für die Einfache Kennzeichnung von Winkeln
\usetkzobj{all}			% Option für 'tkz-euclide'

% PGF (auch in 'export*' benötigt - standalone-Paket-Datei)
\usepgflibrary{%
    plotmarks,		% Scatterplots: Kreise statt Linien
}

% PGFPlots (auch in 'export*' benötigt - standalone-Paket-Datei)
\usepackage{pgfplots}		% Paket zum Plotten laden
\usepgfplotslibrary{%
    external, 		% Bibliothek für das externalisieren von PGFPlots
    statistics,		% Box Plots
    groupplots,		% Groupplots
    smithchart,     % Smith Charts
    polar,          % polar plots
    fillbetween,    % fill area with colour
}

\usepackage{circuitikz}

% Define the Uni Ulm colors
\definecolor{uniblue}{RGB}{125,154,170}
\definecolor{unired}{RGB}{163,38,56}
%
\tikzset{%
    % arrow shape: stealth, latex
    >=stealth,
    %
    %% flow charts
    % -> rectangle blue
    fcrecb/.style={rectangle, rounded corners, align=center, draw=blue, fill=blue!15},
    % -> ellipse red
    fcelr/.style={ellipse, align = center, draw=red, fill=red!20},
    % -> use in document as
    % \tikzset{%
    %   fcblock/.style={fcrecb, minimum width=4cm, minimum height=1.25cm},
    %   fcdecision/.style={fcelr, minimum width=6cm, minimum height=1.25cm},
    % }
}
%
\pgfplotscreateplotcyclelist{roos}{%
  % Usage: Add the following in th axis environment
  % cycle list name = roos,
  % no markers,
  % mark repeat={4},  % Markers only at every nth Datapoint
  % \addplot+
  {blue, solid, mark=x, mark options={solid}},
  {red, densely dashed, mark=o, mark options={solid}},
  {orange, densely dashdotted, mark=triangle, mark options={solid}},
  {black, densely dotted, mark=square, mark options={solid}},
  {brown, loosely dashed, mark=Mercedes star, mark options={solid}},
}
%
\pgfplotscreateplotcyclelist{roosthick}{%
  % Usage: Add the following in th axis environment
  % cycle list name = roos,
  % no markers,
  % mark repeat={4},  % Markers only at every nth Datapoint
  % \addplot+
  {blue, solid, thick, mark=x, mark options={solid}},
  {red, densely dashed, thick, mark=o, mark options={solid}},
  {orange, densely dashdotted, thick, mark=triangle, mark options={solid}},
  {black, densely dotted, thick, mark=square, mark options={solid}},
  {brown, loosely dashed, thick, mark=Mercedes star, mark options={solid}},
}
%
\ifcase \LANGUAGE
\else
% For german documents the decimal seperator is a comma and not a dot. Set this globally
% http://tex.stackexchange.com/a/247423
\pgfplotsset{every linear axis/.append style={%
    /pgf/number format/use comma,
    /pgf/number format/1000 sep={},
  }
}
\fi
%
% Define the parula colormap
\pgfplotsset{% http://tex.stackexchange.com/questions/228197/
    colormap={parula}{%
        rgb = (0.2081,0.1663,0.5292)
        rgb = (0.2116,0.1898,0.5777)
        rgb = (0.2123,0.2138,0.627)
        rgb = (0.2081,0.2386,0.6771)
        rgb = (0.1959,0.2645,0.7279)
        rgb = (0.1707,0.2919,0.7792)
        rgb = (0.1253,0.3242,0.8303)
        rgb = (0.0591,0.3598,0.8683)
        rgb = (0.0117,0.3875,0.882)
        rgb = (0.006,0.4086,0.8828)
        rgb = (0.0165,0.4266,0.8786)
        rgb = (0.0329,0.443,0.872)
        rgb = (0.0498,0.4586,0.8641)
        rgb = (0.0629,0.4737,0.8554)
        rgb = (0.0723,0.4887,0.8467)
        rgb = (0.0779,0.504,0.8384)
        rgb = (0.0793,0.52,0.8312)
        rgb = (0.0749,0.5375,0.8263)
        rgb = (0.0641,0.557,0.824)
        rgb = (0.0488,0.5772,0.8228)
        rgb = (0.0343,0.5966,0.8199)
        rgb = (0.0265,0.6137,0.8135)
        rgb = (0.0239,0.6287,0.8038)
        rgb = (0.0231,0.6418,0.7913)
        rgb = (0.0228,0.6535,0.7768)
        rgb = (0.0267,0.6642,0.7607)
        rgb = (0.0384,0.6743,0.7436)
        rgb = (0.059,0.6838,0.7254)
        rgb = (0.0843,0.6928,0.7062)
        rgb = (0.1133,0.7015,0.6859)
        rgb = (0.1453,0.7098,0.6646)
        rgb = (0.1801,0.7177,0.6424)
        rgb = (0.2178,0.725,0.6193)
        rgb = (0.2586,0.7317,0.5954)
        rgb = (0.3022,0.7376,0.5712)
        rgb = (0.3482,0.7424,0.5473)
        rgb = (0.3953,0.7459,0.5244)
        rgb = (0.442,0.7481,0.5033)
        rgb = (0.4871,0.7491,0.484)
        rgb = (0.53,0.7491,0.4661)
        rgb = (0.5709,0.7485,0.4494)
        rgb = (0.6099,0.7473,0.4337)
        rgb = (0.6473,0.7456,0.4188)
        rgb = (0.6834,0.7435,0.4044)
        rgb = (0.7184,0.7411,0.3905)
        rgb = (0.7525,0.7384,0.3768)
        rgb = (0.7858,0.7356,0.3633)
        rgb = (0.8185,0.7327,0.3498)
        rgb = (0.8507,0.7299,0.336)
        rgb = (0.8824,0.7274,0.3217)
        rgb = (0.9139,0.7258,0.3063)
        rgb = (0.945,0.7261,0.2886)
        rgb = (0.9739,0.7314,0.2666)
        rgb = (0.9938,0.7455,0.2403)
        rgb = (0.999,0.7653,0.2164)
        rgb = (0.9955,0.7861,0.1967)
        rgb = (0.988,0.8066,0.1794)
        rgb = (0.9789,0.8271,0.1633)
        rgb = (0.9697,0.8481,0.1475)
        rgb = (0.9626,0.8705,0.1309)
        rgb = (0.9589,0.8949,0.1132)
        rgb = (0.9598,0.9218,0.0948)
        rgb = (0.9661,0.9514,0.0755)
        rgb = (0.9763,0.9831,0.0538)
    }
}
 % Load default set of graphic settings
\pgfplotsset{compat=newest}	% Do not use compatibility mode, u.a. für matlab2tikz
%\pgfplotsset{plot coordinates/math parser=false} % Für matlab2TikZ

% Kein externalize: https://tex.stackexchange.com/questions/23587/

%%% +++---~~~ Anfang: Dokument ~~~---+++ %%%%%%%%%%%%%%%%%%%%%%%%%%%%%%%%%%%%%%%%%%%%%%%%%%%%%%%%%%%
\begin{document}		% Keine Leeren Zeilen nach '\begin{document}', S. 7
%%
% Für Fragen wenden an
% Fabian Roos, fabian.roos@uni-ulm.de / studium@froos.de
%%
\begin{tikzpicture}
  % axis
  \draw[->] (0,0) -- (10,0) node[below] {$t$};
  \draw[->] (0,0) -- (0,2.0) node[left] {$f$};

  % -> label f_0
  \draw (0.1,0.5) -- (-0.1,0.5) node[left] {$f_{\text{0}}$};


  %% first block
  % draw chirps
  \foreach \x in {0, 0.75, 1.5}
  {%
    \draw[blue] (\x,0) -- ({\x+0.5},1);
  }
  % draw the box
  \draw[very thin, dashed, gray] (0,0) -- (0,1) -- (2.25,1) -- (2.25,0);
  % label
  \node[above] at ({2.25/2},1) {Block $k$};


  %% first repetition
  \begin{scope}[shift={(6.75,0)}]
    %% first block
    % draw chirps
    \foreach \x in {0, 0.75, 1.5}
    {%
      \draw[blue] (\x,0) -- ({\x+0.5},1);
    }
    % draw the box
    \draw[very thin, dashed, gray] (0,0) -- (0,1) -- (2.25,1) -- (2.25,0);
    % label
    \node[above] at ({2.25/2},1) {Block $k+1$};
  \end{scope}
\end{tikzpicture}

%%
% Für Fragen wenden an
% Fabian Roos, fabian.roos@uni-ulm.de / studium@froos.de
%%

% Plotting a Function
% \begin{tikzpicture}
%   \begin{axis}[
%       % enlargelimits = false,
%       grid = major,
%       % xmin = 0,
%       % xmax = 3,
%       % xtick = {0,0.5,1,...,3},
%       title = {In Document Use Caption Instead Of Title},
%       xlabel = {$t$ in \si{\micro\second}},
%       ylabel = {Voltage in \si{\volt}},
%     ]

%     \addplot[
%       blue,
%       domain = -pi:pi,
%       samples = 100,
%     ]{x + sin(deg(x))}; % input of sin() in degree

%   \end{axis}
% \end{tikzpicture}


% Plotting Data
\begin{tikzpicture}
  \begin{axis}[
      % enlargelimits = false,
      grid = major,
      xlabel = {Samples},
      ylabel = {Power in \si{\decibel}},
      % title = {In Document Use Caption Instead Of Title},
      legend pos = north west,
      legend cell align = left,
      % legend columns = -1,
      cycle list name = roos,
      no markers,
    ]

    % if brackets are used, the cycle list is deativated. Activate it with a +
    \addplot+[
    ]
    table[col sep = comma]{gfx/example_plots.csv};

    % no brackets uses the cycle list
    \addplot
    table[x index = 0, y index = 2, col sep = comma]{gfx/example_plots.csv};

    \addlegendentry{simulation};
    \addlegendentry{measurement};
  \end{axis}
\end{tikzpicture}


% % Plotting Data: Scatter Plots
% \begin{tikzpicture}
%   \begin{axis}
%     [
%       % enlargelimits = false,
%       grid = major,
%       xlabel = {Samples},
%       ylabel = {Power in \si{\decibel}},
%       % title = {In Document Use Caption Instead Of Title},
%       % colormap/jet,
%       colorbar,
%     ]
%     \addplot[
%       scatter,
%       scatter src = explicit,
%       only marks,
%     ]
%     table[x index = 0, y index = 1, meta index = 2, col sep = comma]{gfx/example_plots.csv};
%   \end{axis}
% \end{tikzpicture}


% % Plotting Data: Scatter Plot with variable marker colour AND size
% % Data Structure used
% % -> 1. column: x values
% % -> 2. column: y values
% % -> 3. column: colour of markers
% % -> 4. column: size of markers
% % Source: Using the size: http://tex.stackexchange.com/questions/98646/
% % Source: No need for nameing the columns because of \thisrowno{}
% %         http://tex.stackexchange.com/questions/41033/
% \begin{tikzpicture}
%   \begin{axis}[
%       % enlargelimits = false,
%       grid = major,
%       xlabel = {Samples},
%       ylabel = {Power in \si{\decibel}},
%       colorbar,
%     ]

%     \addplot[
%       scatter,
%       only marks,
%       scatter src=explicit,
%       visualization depends on=\thisrowno{3}\as\wtwo,
%       scatter/@pre marker code/.append style={%
%         /tikz/mark size=\wtwo
%       }
%     ]
%     table[x index=0, y index=2, meta index=2, col sep=comma]{gfx/example_plots.csv};
%   \end{axis}
% \end{tikzpicture}

\begin{tikzpicture}
  \begin{axis}[
      axis on top,
      enlargelimits = false,
      colorbar,
      colorbar style = {%
        ylabel = {Amplitude in \si{\decibel}},
        % ytick = {-50,-40,...,0},
      },
      point meta min = -51.1991,
      point meta max = 0,
      % xtick = {0,20,...,100},
      % ytick = {0,20,...,60},
      xlabel = {$R$ in \si{\metre}},
      ylabel = {$v$ in \si{\metre\per\second}},
      % xmin = 10,
      % xmax = 20,
      % ymin = -30,
      % ymax = 30,
    ]
    \addplot graphics
    [xmin = 0, xmax = 51, ymin = -36.055, ymax = 35.4916]
    {gfx/imagesc-1.png};
  \end{axis}
\end{tikzpicture}

\begin{tikzpicture}
  \begin{axis}[
      grid,
      minor tick num = 1,
      zmin = -46.7717,
      zmax = 0,
      xtick = {0,15,...,45},
      ytick = {-30,-15,...,30},
      ztick = {-45,-30,...,0},
      xlabel = {$R$ in \si{\metre}},
      ylabel = {$v$ in \si{\metre\per\second}},
      colorbar,
      colorbar style = {%
        % ytick = {-50,-40,...,0},
        ylabel = {Amplitude in \si{\decibel}},
      },
      point meta min = -46.7717,
      xmin=0,
      point meta max = 0,
    ]
    \addplot3 graphics[
      points={% important
        (15.2,29.86,0) => (73,239-0)
        (0.2,35.49,-30.1) => (1,239-140)
        (1.8,-35.49,-39.25) => (205,239-239)
        (50.8,-36.06,-31.12) => (389,239-148)
    }]{gfx/surf3D.png};
  \end{axis}
\end{tikzpicture}

\tikzset{%
	% Self defined bulding blocks. 
	% Nevertheless circutikz has implemented filters, couplers and other components since version 0.4, they are mostly implemented as bipoles.
	% The usage of bipoles: \draw (start) to[lowpass/amp/adc,....] (end).
	% The problem is, that if one wants to use arrows, the arrows in bipoles can not be sat manual (fixed in circuitikz source) AND THEY ARE NOT CONSISTENT
	% Also it's quite a mess, which component is a monopole, simple block, bipol, quad/triple etc
	% Following are a few examples on how to define your own blocks. 
	%
	% % % % % % % % % % % % % % % % % % % % % % % % % % % % % % % % % % % % % % % % % % % % % % % % % % % % % % % % % % % % % % % % % % % % %
	% % % % % % % % % % % % % % % % % % % % % % % % % % % % % % % % % % % % % % % % % % % % % % % % % % % % % % % % % % % % % % % % % % % % %
	% % % % % % % % % % % % % % % % % % % % % % % % % % % % % % % % % % % % % % % % % % % % % % % % % % % % % % % % % % % % % % % % % % % % %
	% % % % % % % % % % % % % % % % % % % % % % % % % % % % % % % % % % % % % % % % % % % % % % % % % % % % % % % % % % % % % % % % % % % % %
	%
	% Standard block definition, the width and height is adopted from the circutizk source code, so don't mind the strange values. Also the linewidth is set according to the circutrikz source code.
	block/.style    	= 	{draw, fill=white, thick, rectangle, minimum height = 0.98cm, minimum width = 0.98cm, node distance=2.5cm, line width=1.5pt},
	%
	% Standard circular block
	circleblock/.style	= 	{draw, fill=white, thick, circle, minimum width = 0.98cm,  line width=1.5pt, node distance=2.5cm},
	%
	% Label for circuitikz nodes, as they're reference is in the middle and not on the outer edge of the node....
	circuitikzlabel/.style	=	{label={[label, label distance=0.5cm]#1}},
	%
	%
	%
	% VCO/Oscillator 
	myVCO/.style			= 	{circleblock, path picture={%
		\draw[line width=0.75pt] 	($(path picture bounding box.west)+(0.09cm,0)$) sin ($(path picture bounding box.center)-(0.2cm,-0.2cm)$) cos  (path picture bounding box.center) sin ($(path picture bounding box.center)-(-0.2cm,0.2cm)$) cos ($(path picture bounding box.east)-(0.09cm,0)$);
		}
	},
	% Amplifier, as circuitikz does only provite amplifiers as 2-ports/bipoles
	myAMP/.style		= 	{block, node distance=2.5cm, path picture={%
		\draw[fill=white, line width=0.75pt] ($(path picture bounding box.center)+(0.7em,0)$) -- ($(path picture bounding box.center)-(0.7em,-0.7em)$) -- ($(path picture bounding box.center)-(0.7em,0.7em)$)  -- cycle;
		}
	},
	% Same for ADC
	myADC/.style 	=	{block, path picture={%
		\draw[line width=0.75pt] 	(path picture bounding box.south west) -- (path picture bounding box.north east);
		\node[] at ($(path picture bounding box.center)+(-.5em,.5em)$) () {D};
		\node[] at ($(path picture bounding box.center)+(.5em,-.5em)$) () {A};
		} 
	},
	% Same for filters
	myLP/.style	=	{block, path picture={%
		%Sine-Waves
		\draw[line width=.75pt] 	($(path picture bounding box.west)+(0.3em,0)$) sin ($(path picture bounding box.center)-(0.50em,-0.3em)$) cos  (path picture bounding box.center) sin ($(path picture bounding box.center)-(-0.50em,0.3em)$) cos ($(path picture bounding box.east)-(0.3em,0)$);
		\draw[line width=0.75pt] 	($(path picture bounding box.west)+(0.3em,-0.65em)$) sin ($(path picture bounding box.center)-(0.50em,0.35em)$) cos  ( $(path picture bounding box.center)-(0,0.65em)$) sin ($(path picture bounding box.center)-(-0.50em,0.95em)$) cos ($(path picture bounding box.east)-(0.3em,0.65em)$);
		\draw[line width=0.75pt] 	($(path picture bounding box.west)+(0.3em,0.65em)$) sin ($(path picture bounding box.center)-(0.50em,-0.95em)$) cos  ( $(path picture bounding box.center)+(0,0.65em)$) sin ($(path picture bounding box.center)-(-0.50em,-0.35em)$) cos ($(path picture bounding box.east)-(0.3em,-0.65em)$);
		% Cancelation
		\draw[line width=0.75pt] 	($(path picture bounding box.center)-(0.2em,0.2em)$) -- (path picture bounding box.center) -- ($(path picture bounding box.center)+(0.2em,0.2em)$) ;
		\draw[line width=0.75pt] 	($(path picture bounding box.center)-(0.2em,-0.45em)$) -- ($(path picture bounding box.center)+(0,0.65em)$) -- ($(path picture bounding box.center)+(0.2em,0.85em)$) ;
		}
	},
}
\begin{tikzpicture}[line width=0.7pt,>=latex,node distance=2.5cm]
	% First: All building blocks are placed relative to the first component
	\draw (0,0)
		node[myVCO, label={below:VCO}] (oszi) {}
		% As the coupler ports are not in the middle, based on the size (again extraceted from circutikz source code), an yshift is perfomed to have the input on the same height as the output of the oszillator. The xshift is used to place the VCO and ADC on the same y-value after all.
		node[coupler, right of=oszi, yshift=-0.455cm, xshift=1cm] (coupler) {}
		% Undo the yshift as the output of the coupler is on the same height as the input of the amplifier
		node[myAMP, right of=coupler, yshift=0.455cm, node distance=4cm, label={below:PA}] (pa) {}
		% Circulator is rotated that the ports are on the correct position, normally ports are arranged as follows:
		% 
		%				  o------------o
		%						|
		%						o
		node[circulator, below right of=pa, rotate=90] (circ) {}
		node[antenna, right of = circ] (antenna) {}
		node[myAMP, below left of=circ, rotate=180, label={LNA}] (lna) {}
		% Used redefinition of label, otherwise the label would be overlapping with the mixer shape
		node[mixer, left of=lna, circuitikzlabel={below:mixer}] (mixer) {}
		node[myLP, left of=mixer, label={below:lowpass}] (lowpass) {}
		node[myADC, left of=lowpass] (adc) {};
	
	% Connect everything together
	\draw[->] (oszi) -- (coupler.4);
	\draw[->] (coupler.3) -- (pa.west);
	% Match is placed relative to coupler port, yscale=-1 mirrors the component at the y-axis
	\draw[] (coupler.1) -| ++(-0.5cm,-0.1cm) node[match, rotate=-90, yscale=-1] {};
	\draw[->] (pa.east) -| (circ.2);
	\draw[-] (circ.3) -- (antenna);
	\draw[->] (circ.1) |- (lna.west);
	\draw[->] (lna.east) -- (mixer.east);
	\draw[->] (coupler.2) -| (mixer.north);
	\draw[->] (mixer.west)  -- (lowpass.east);
	\draw[->] (lowpass.west) -- (adc.east);
\end{tikzpicture}

% Smith Chart Example
% by Fabian Roos (Plotting) and Philipp Hügler (RF matching)
% 29.02.2016

\begin{tikzpicture}
  \begin{smithchart}
    % start at point (0.2 + 0.5i) and mirror the point to (0.6897 - 1.7241i)
    % -> calculate the mirrored point by 1/(...)
    \addplot[
      gray,
      dashed,
      ->,
      mark=x,
      mark options={black,solid},
    ]
    coordinates{%
      (0.2,0.5)
      (0.6897,-1.7241)
    };

    % rotate with a parallel C
    % -> start from 1/(...) and add 1i*omega*C
    % -> calculate points and import them from a csv file
    \addplot[
      red,
      ->,
      no markers,
    ]
    table[col sep = comma]{gfx/smith_chart.csv};

    % mirror the endpoint
    \addplot[
      gray,
      dashed,
      ->,
      mark=x,
      mark options={black,solid},
    ]
    coordinates{%
      (0.6897,0.4624) (1.0003,-0.6707)
    };

    % rotate with a serial L
    % -> start from the endpoint of the prvious rotation and add 1i*omega*L
    % -> calculate points and import them from a csv file
    \addplot[
      blue,
      ->,
      no markers,
    ]
    table[col sep = comma, x index = 2, y index = 3]{gfx/smith_chart.csv};
  \end{smithchart}
\end{tikzpicture}

% created with the following Matlab Script
% -> predefined C and L values with Simulator by Hügler
% %%%%%%%%%%%%%%%%%%%%%%%%%%%%%%%%%%%%%%%%%%%%%%%%%%%%%%%%%%%%%%%%%%%%%%%%%%%
% %%% Fabian Roos, fabian.roos@uni-ulm.de %%%%%%%%%%%%%%%%%%%%%%%%%%%%%%%%%%%
% %%%                                                                     %%%
% %%% Matlab Script for Obtaining Coordinates                             %%%
% %%%                                                                     %%%
% %%%%%%%%%%%%%%%%%%%%%%%%%%%%%%%%%%%%%%%%%%%%%%%%%%%%%%%%%%%%%%%%%%%%%%%%%%%
% %%% Revisions:                                                          %%%
% %%% 29.02.16: * basic version                                           %%%
% %%%%%%%%%%%%%%%%%%%%%%%%%%%%%%%%%%%%%%%%%%%%%%%%%%%%%%%%%%%%%%%%%%%%%%%%%%%
% clear
% clc
% 
% % define the omega
% omega = 2*pi*10e9;
% 
% % this is the start impedance value
% imp = 0.2+0.5i;
% 
% % the two precalculated target values for the rotation
% C = linspace(0, 696e-15*50); % admitance plane: * 50
% L = linspace(0, 534e-12/50); % impedance plane: / 50
% 
% % the starting impedance is the mirrord one (admitance plane)
% % -> and then use i*omega*C (admitance plane!)
% imp_para_c = 1/imp + (1i*omega*C);
% 
% % the end point is mirrored end point of the C rotation
% % -> and then use i*omega*L (impedance plane!)
% imp_seriel_L = 1/(imp_para_c(end)) + 1i * omega * L;
% 
% % save the two rotations with the real and imaginary parts
% exp_vec = [real(imp_para_c).', imag(imp_para_c).', real(imp_seriel_L).' imag(imp_seriel_L).'];
% csvwrite('smith_chart.csv', exp_vec)
% system('mv smith_chart.csv ~/Documents/wip/')

% Example written by Fabian Roos, 08.2016
% basic idea:
% to ensure that all float charts use the same layout a common style is defined in my (Fabian Roos)
% style definition file 'gfx/rs_tikz.tex' which is loaded. Because the width of the boxes is
% dependant from the content no dimensions are set in the style file. These values need to be given
% in your current document. This is done here in the header at the 'tikzset' position. The basic
% style is inherited by the style file. Only the dimensions of the minimum width and height are
% specified.
% The minimum minimum width and height of elements are automatically calculated by TikZ to ensure
% good looking nodes. The problem is, that with a changing text the nodes would have different
% sizes. Therefore the minimum width and height are specified.
% The same holds for the node distance. This is the distance which is used to place to nodes next to
% each other using the 'right' / 'letf' / 'below' / 'above' of command. This distance is calculated
% from the middle point to the middle point. Because the general node distance would also apply to
% the vertical shift, this shift is specified again if a vertical placement is used.
% The rest is normal node placement.

\begin{tikzpicture}[node distance=4cm]
  \tikzset{%
    fcblock/.style={fcrecb, minimum width=2.5cm, minimum height=1.00cm},
    fcdecision/.style={fcelr, minimum width=4cm, minimum height=1.00cm},
  }

  % shift in y direction to compensate node distance in vertical direction
  \newcommand{\ys}{2.5}

  % first row
  \node[fcblock] (first) at (0,0) {first step};

  % second row
  \node[fcblock, yshift=\ys cm, below of=first] (second) {second step};
  \node[fcdecision, right of=second] (decide) {what to do?};

  % third row
  \node[fcblock, yshift=\ys cm, below of=second] (third) {third step};
  \node[fcblock, yshift=\ys cm, below of=decide] (quit) {quit!};

  % connections
  \draw[->] (first)         --  (second);
  \draw[->] (second)        --  (decide);
  \draw[->] (decide.south)  --  (third.north);
  \draw[->] (decide)        --  (quit);
\end{tikzpicture}

\end{document}
%%% +++---~~~ Ende: Dokument ~~~---+++ %%%%%%%%%%%%%%%%%%%%%%%%%%%%%%%%%%%%%%%%%%%%%%%%%%%%%%%%%%%%%


%%%%%%%%%%%%%%%%%%%%%%%%%%%%%%%%%%%%%%%%%%%%%%%%%%%%%%%%%%%%%%%%%%%%%%%%%%%%%%%%%%%%%%%%%%%%%%%%%%%%
%%% +++---~~~~ Ende: Vorlage für Grafikexport für Speicherung als Bild ~~~---+++ %%%%%%%%%%%%%%%%%%%
%%%%%%%%%%%%%%%%%%%%%%%%%%%%%%%%%%%%%%%%%%%%%%%%%%%%%%%%%%%%%%%%%%%%%%%%%%%%%%%%%%%%%%%%%%%%%%%%%%%%
