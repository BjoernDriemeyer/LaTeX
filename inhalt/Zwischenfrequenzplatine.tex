\chapter{Zwischenfrequenzplatine}
In diesem Kapitel wird die Signalkonditionierung beschrieben. Dabei wird die Signalkonditionierung auf einer seperaten Platine, die Zwischenfrequenzplatine, vorgenommen. Eingangssignal werden zwei Kanäle, I und Q, in jeweils differentieller Form sein. Diese werden direkt vom Radarchip auf das Board geführt. Ziel ist es, als Ausgangssignal ein für die ADC Wandlung konditioniertes Signal zu erhalten. Ebenfalls soll die Zwischenfrequenzplatine unabhängig vom verwendeten Radarchip sein. 
\section{Anforderung und Verwendung}
Ein wesentliches Kriterium ist die erwartete Bandbreite. Diese ist abhängig von Radarchip und den darin verwendeten Mischern. Für den primär verwendeten \SI{122}{G\hertz} Chip ist eine Zwischenfrequenzbandbreite von 0-\SI{20}{M\hertz} angegeben. Dabei ist die Ausgangsimpedanz ausgehend vom Radarchip \SI{500}{\ohm}. Weiterhin muss der Eingangspegel bestimmt werden. Minimal liegt bei nach (LINK BUDGET REFERENZ) bei \SI{-44}{dBm}.
\section{Auswahl und Anordnung der Komponenten}
In Bild \ref{fig:IFBoard_Block} ist das Blockdiagramm der Zwischenfrequenzplatine dargestellt. Der Eingang links im Bild ist das vom Radarchip ausgehende Signal. Der Ausgang rechts ist das an den ADC weitergegebene, signalkonditionierte Signal. Die ersten drei Systemkomponenten sind differentiell zu differentiell. Demnach besitzen sie jeweils pro Kanal zwei Eingänge und zwei Ausgänge. Die letzte Systemstufe hat einen differentiellen Eingang und einen einfachen Ausgang.\\
Als erste Komponente wurde ein ADA4937-2 von Analog Devices gewählt. Dieser differentiell zu differentiell Verstärker wird mit \SI{5}{\volt} betrieben. Der ADA4837-2 hat eine hohe differentielle Eingangsimpedanz von \SI{6}{M\ohm} sowie eine auf den Gleichanteil bezogene Eingangsimpedanz von \SI{3}{M\ohm}. Dies sorgt für Stabilität bei den ausgehenden Radarchipsignalen. Zusätzlich ist die Gleichtaktunterdrückung \SI{-80}{dB} wodurch Rauschen auf dem Gleichtakt stets unterdrückt wird. Die \SI{-3}{dB} Bandbreite liegt bei \SI{1.9}{G\hertz} und ist somit mehr als ausreichend. Der Gain kann über ein Widerstandsnetzwerk von vier Widerständen eingestellt werden. \\
\begin{figure}[tbp]
  \centering
  \input{gfx/IfBoard_Block.tikz}
  \caption{Blockdiagramm Ziwschenfrequenzplatine}
  \label{fig:IFBoard_Block}
\end{figure} 
Die zweite Komponente ist ein HMC960 von Analog Devices. Dieser in Kombination mit der dritten Komponente, einem HMC900, ist die eigentliche Signalkonditionierung bestehend aus einem Variablen-Gain-Verstärker (HMC960) sowie einem Variablen-Tiefpass (HMC900). Der HMC900 hat dabei noch eine optionale \SI{10}{dB} Gain Stufe.\\
Die differentiell zu einfach Konvertierung wird von einem AD8034 vorgenommen. Dieser ist die vierte und letzte Systemkomponente.
\section{Filterentwurf}
Um Rauschen auf den Gleichanteil zu unterdrücken werden zwischen den einzelnen Systemkomponenten Filter eingefügt, die bei Bedarf bestückt werden können. Zusätzlich soll das eigentliche Signal nicht gedämpft werden. Die Bandbreite von \SI{20}{M\hertz} muss demnach eingehalten werden.
\begin{figure}[tbp]
  \centering
  \tikzset{%
	% Self defined bulding blocks. 
	% Nevertheless circutikz has implemented filters, couplers and other components since version 0.4, they are mostly implemented as bipoles.
	% The usage of bipoles: \draw (start) to[lowpass/amp/adc,....] (end).
	% The problem is, that if one wants to use arrows, the arrows in bipoles can not be sat manual (fixed in circuitikz source) AND THEY ARE NOT CONSISTENT
	% Also it's quite a mess, which component is a monopole, simple block, bipol, quad/triple etc
	% Following are a few examples on how to define your own blocks. 
	%
	% % % % % % % % % % % % % % % % % % % % % % % % % % % % % % % % % % % % % % % % % % % % % % % % % % % % % % % % % % % % % % % % % % % % %
	% % % % % % % % % % % % % % % % % % % % % % % % % % % % % % % % % % % % % % % % % % % % % % % % % % % % % % % % % % % % % % % % % % % % %
	% % % % % % % % % % % % % % % % % % % % % % % % % % % % % % % % % % % % % % % % % % % % % % % % % % % % % % % % % % % % % % % % % % % % %
	% % % % % % % % % % % % % % % % % % % % % % % % % % % % % % % % % % % % % % % % % % % % % % % % % % % % % % % % % % % % % % % % % % % % %
	%
	% Standard block definition, the width and height is adopted from the circutizk source code, so don't mind the strange values. Also the linewidth is set according to the circutrikz source code.
	block/.style    	= 	{draw, fill=white, thick, rectangle, minimum height = 0.98cm, minimum width = 0.98cm, node distance=2.5cm, line width=1.5pt},
	%
	% Standard circular block
	circleblock/.style	= 	{draw, fill=white, thick, circle, minimum width = 0.98cm,  line width=1.5pt, node distance=2.5cm},
	%
	% Label for circuitikz nodes, as they're reference is in the middle and not on the outer edge of the node....
	circuitikzlabel/.style	=	{label={[label, label distance=0.5cm]#1}},
	%
	%
	%
	% VCO/Oscillator 
	myVCO/.style			= 	{circleblock, path picture={%
		\draw[line width=0.75pt] 	($(path picture bounding box.west)+(0.09cm,0)$) sin ($(path picture bounding box.center)-(0.2cm,-0.2cm)$) cos  (path picture bounding box.center) sin ($(path picture bounding box.center)-(-0.2cm,0.2cm)$) cos ($(path picture bounding box.east)-(0.09cm,0)$);
		}
	},
	% Amplifier, as circuitikz does only provite amplifiers as 2-ports/bipoles
	myAMP/.style		= 	{block, node distance=2.5cm, path picture={%
		\draw[fill=white, line width=0.75pt] ($(path picture bounding box.center)+(0.7em,0)$) -- ($(path picture bounding box.center)-(0.7em,-0.7em)$) -- ($(path picture bounding box.center)-(0.7em,0.7em)$)  -- cycle;
		}
	},%%
	% Block	
	myBlock/.style    	= 	{draw, fill=white, thick, rectangle, minimum height = 0.98cm, minimum width = 0.98cm, node distance=2.5cm, line width=1.5pt},
	myBigBlock/.style    	= 	{draw, fill=white, thick, rectangle, minimum height = 0.98cm, minimum width = 2.94cm, node distance=2.5cm, line width=1.5pt},	
	% Same for ADC
	myADC/.style 	=	{block, path picture={%
		\draw[line width=0.75pt] 	(path picture bounding box.south west) -- (path picture bounding box.north east);
		\node[] at ($(path picture bounding box.center)+(-.5em,.5em)$) () {D};
		\node[] at ($(path picture bounding box.center)+(.5em,-.5em)$) () {A};
		} 
	},
	% Same for filters
	myLP/.style	=	{block, path picture={%
		%Sine-Waves
		\draw[line width=.75pt] 	($(path picture bounding box.west)+(0.3em,0)$) sin ($(path picture bounding box.center)-(0.50em,-0.3em)$) cos  (path picture bounding box.center) sin ($(path picture bounding box.center)-(-0.50em,0.3em)$) cos ($(path picture bounding box.east)-(0.3em,0)$);
		\draw[line width=0.75pt] 	($(path picture bounding box.west)+(0.3em,-0.65em)$) sin ($(path picture bounding box.center)-(0.50em,0.35em)$) cos  ( $(path picture bounding box.center)-(0,0.65em)$) sin ($(path picture bounding box.center)-(-0.50em,0.95em)$) cos ($(path picture bounding box.east)-(0.3em,0.65em)$);
		\draw[line width=0.75pt] 	($(path picture bounding box.west)+(0.3em,0.65em)$) sin ($(path picture bounding box.center)-(0.50em,-0.95em)$) cos  ( $(path picture bounding box.center)+(0,0.65em)$) sin ($(path picture bounding box.center)-(-0.50em,-0.35em)$) cos ($(path picture bounding box.east)-(0.3em,-0.65em)$);
		% Cancelation
		\draw[line width=0.75pt] 	($(path picture bounding box.center)-(0.2em,0.2em)$) -- (path picture bounding box.center) -- ($(path picture bounding box.center)+(0.2em,0.2em)$) ;
		\draw[line width=0.75pt] 	($(path picture bounding box.center)-(0.2em,-0.45em)$) -- ($(path picture bounding box.center)+(0,0.65em)$) -- ($(path picture bounding box.center)+(0.2em,0.85em)$) ;
		}
	},
}
\begin{tikzpicture}[line width=0.7pt,>=latex,node distance=2.5cm,scale = 0.8]
	% First: All building blocks are placed relative to the first component
	
	\draw (0,0) to[R=$R_1$] (2,0);	
	\draw (0,-2) to[R=$R_2$] (2,-2);	
	\draw (3,2) to[C=$C_1$] (3,0);	
	\draw (3,-2) to[C=$C_2$] (3,-4);	
	\draw (4,0) to[C=$C_3$] (4,-2);

	\draw[o-](-2,0)--(0,0);
	\draw[o-](-2,-2)--(0,-2);
	
	\draw[-o] (2,0)--(6,0);
	\draw[-o] (2,-2)--(6,-2);
	
	\draw(3,2)node[ground,rotate = 180](){};
	\draw(3,-4)node[ground](){};
\end{tikzpicture}

  \caption{Blockdiagramm CM-Filter}
  \label{fig:CM_Filter_Block}
\end{figure} 
\section{Schaltplan-Erstellung und PCB-Design}
\begin{figure}[tbp]
  \centering
  \input{gfx/CM_Filter_Simulation.tikz}
  \caption{Blockdiagramm CM-Filter}
  \label{fig:CM_Filter_Block}
\end{figure} 
\section{Softwarekonzept}



